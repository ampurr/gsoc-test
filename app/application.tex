% Options for packages loaded elsewhere
\PassOptionsToPackage{unicode}{hyperref}
\PassOptionsToPackage{hyphens}{url}
\PassOptionsToPackage{dvipsnames,svgnames,x11names}{xcolor}
%
\documentclass[
]{article}

\usepackage{amsmath,amssymb}
\usepackage{iftex}
\ifPDFTeX
  \usepackage[T1]{fontenc}
  \usepackage[utf8]{inputenc}
  \usepackage{textcomp} % provide euro and other symbols
\else % if luatex or xetex
  \usepackage{unicode-math}
  \defaultfontfeatures{Scale=MatchLowercase}
  \defaultfontfeatures[\rmfamily]{Ligatures=TeX,Scale=1}
\fi
\usepackage{lmodern}
\ifPDFTeX\else  
    % xetex/luatex font selection
  \setmainfont[]{Roboto Condensed}
  \setmonofont[]{Roboto Mono}
\fi
% Use upquote if available, for straight quotes in verbatim environments
\IfFileExists{upquote.sty}{\usepackage{upquote}}{}
\IfFileExists{microtype.sty}{% use microtype if available
  \usepackage[]{microtype}
  \UseMicrotypeSet[protrusion]{basicmath} % disable protrusion for tt fonts
}{}
\makeatletter
\@ifundefined{KOMAClassName}{% if non-KOMA class
  \IfFileExists{parskip.sty}{%
    \usepackage{parskip}
  }{% else
    \setlength{\parindent}{0pt}
    \setlength{\parskip}{6pt plus 2pt minus 1pt}}
}{% if KOMA class
  \KOMAoptions{parskip=half}}
\makeatother
\usepackage{xcolor}
\setlength{\emergencystretch}{3em} % prevent overfull lines
\setcounter{secnumdepth}{-\maxdimen} % remove section numbering
% Make \paragraph and \subparagraph free-standing
\ifx\paragraph\undefined\else
  \let\oldparagraph\paragraph
  \renewcommand{\paragraph}[1]{\oldparagraph{#1}\mbox{}}
\fi
\ifx\subparagraph\undefined\else
  \let\oldsubparagraph\subparagraph
  \renewcommand{\subparagraph}[1]{\oldsubparagraph{#1}\mbox{}}
\fi


\providecommand{\tightlist}{%
  \setlength{\itemsep}{0pt}\setlength{\parskip}{0pt}}\usepackage{longtable,booktabs,array}
\usepackage{calc} % for calculating minipage widths
% Correct order of tables after \paragraph or \subparagraph
\usepackage{etoolbox}
\makeatletter
\patchcmd\longtable{\par}{\if@noskipsec\mbox{}\fi\par}{}{}
\makeatother
% Allow footnotes in longtable head/foot
\IfFileExists{footnotehyper.sty}{\usepackage{footnotehyper}}{\usepackage{footnote}}
\makesavenoteenv{longtable}
\usepackage{graphicx}
\makeatletter
\def\maxwidth{\ifdim\Gin@nat@width>\linewidth\linewidth\else\Gin@nat@width\fi}
\def\maxheight{\ifdim\Gin@nat@height>\textheight\textheight\else\Gin@nat@height\fi}
\makeatother
% Scale images if necessary, so that they will not overflow the page
% margins by default, and it is still possible to overwrite the defaults
% using explicit options in \includegraphics[width, height, ...]{}
\setkeys{Gin}{width=\maxwidth,height=\maxheight,keepaspectratio}
% Set default figure placement to htbp
\makeatletter
\def\fps@figure{htbp}
\makeatother


\usepackage{booktabs}
\usepackage{microtype}
\makeatletter
\makeatother
\makeatletter
\makeatother
\makeatletter
\@ifpackageloaded{caption}{}{\usepackage{caption}}
\AtBeginDocument{%
\ifdefined\contentsname
  \renewcommand*\contentsname{Table of contents}
\else
  \newcommand\contentsname{Table of contents}
\fi
\ifdefined\listfigurename
  \renewcommand*\listfigurename{List of Figures}
\else
  \newcommand\listfigurename{List of Figures}
\fi
\ifdefined\listtablename
  \renewcommand*\listtablename{List of Tables}
\else
  \newcommand\listtablename{List of Tables}
\fi
\ifdefined\figurename
  \renewcommand*\figurename{Figure}
\else
  \newcommand\figurename{Figure}
\fi
\ifdefined\tablename
  \renewcommand*\tablename{Table}
\else
  \newcommand\tablename{Table}
\fi
}
\@ifpackageloaded{float}{}{\usepackage{float}}
\floatstyle{ruled}
\@ifundefined{c@chapter}{\newfloat{codelisting}{h}{lop}}{\newfloat{codelisting}{h}{lop}[chapter]}
\floatname{codelisting}{Listing}
\newcommand*\listoflistings{\listof{codelisting}{List of Listings}}
\makeatother
\makeatletter
\@ifpackageloaded{caption}{}{\usepackage{caption}}
\@ifpackageloaded{subcaption}{}{\usepackage{subcaption}}
\makeatother
\makeatletter
\@ifpackageloaded{tcolorbox}{}{\usepackage[skins,breakable]{tcolorbox}}
\makeatother
\makeatletter
\@ifundefined{shadecolor}{\definecolor{shadecolor}{rgb}{.97, .97, .97}}
\makeatother
\makeatletter
\makeatother
\makeatletter
\@ifpackageloaded{sidenotes}{}{\usepackage{sidenotes}}
\@ifpackageloaded{marginnote}{}{\usepackage{marginnote}}
\makeatother
\makeatletter
\makeatother
\ifLuaTeX
  \usepackage{selnolig}  % disable illegal ligatures
\fi
\IfFileExists{bookmark.sty}{\usepackage{bookmark}}{\usepackage{hyperref}}
\IfFileExists{xurl.sty}{\usepackage{xurl}}{} % add URL line breaks if available
\urlstyle{same} % disable monospaced font for URLs
\hypersetup{
  pdftitle={animint2 Documentation and Bug Fix Project: An Application},
  colorlinks=true,
  linkcolor={\#043521},
  filecolor={Maroon},
  citecolor={Blue},
  urlcolor={Blue},
  pdfcreator={LaTeX via pandoc}}

\title{animint2 Documentation and Bug Fix Project: An Application}
\author{}
\date{}

\begin{document}
\maketitle
\ifdefined\Shaded\renewenvironment{Shaded}{\begin{tcolorbox}[frame hidden, boxrule=0pt, breakable, borderline west={3pt}{0pt}{shadecolor}, interior hidden, enhanced, sharp corners]}{\end{tcolorbox}}\fi

\renewcommand*\contentsname{Contents:}
{
\hypersetup{linkcolor=}
\setcounter{tocdepth}{3}
\tableofcontents
}
\emph{There's been an explosion of AI tools this year. I hereby certify
that I haven't used them for this application in any way.}

\hypertarget{project-info}{%
\subsection{Project Info}\label{project-info}}

\textbf{Project Title:} \texttt{animint2} Documentation and Bug Fix
Project

\textbf{Project Short Title:} \texttt{animint2} Documentation Project

\textbf{URL:} For the project that I'm hoping to work on?
\href{https://github.com/rstats-gsoc/gsoc2023/wiki/Animated-interactive-ggplots}{It's
here}.

\hypertarget{proposal-summary}{%
\subsection{Proposal Summary}\label{proposal-summary}}

The \texttt{animint2} Documentation and Bug Fix Project will make the
\texttt{animint2} reference documentation more accessible, fix errors in
the documentation, and reduce the number of bugs by at least one.

\hypertarget{broad-scope-of-the-project}{%
\subsection{Broad Scope of the
Project}\label{broad-scope-of-the-project}}

My goals for this project are as follows:

\begin{enumerate}
\def\labelenumi{\arabic{enumi}.}
\tightlist
\item
  To improve access to \texttt{animint2}'s features for scientists,
  students, and other users; and
\item
  To make \texttt{animint2} easier and more pleasant to use for
  scientists, students, and other users.
\end{enumerate}

This will involve:

\begin{enumerate}
\def\labelenumi{\arabic{enumi}.}
\tightlist
\item
  Improving documentation for \texttt{animint2} via a reference site
  accompanying the
  \href{https://rcdata.nau.edu/genomic-ml/animint2-manual/Ch00-preface.html}{\texttt{animint2}
  Manual}; and
\item
  Documenting and fixing bugs and issues in \texttt{animint2}.
\end{enumerate}

\texttt{animint2} already has a number of useful features for
interactive data visualization. This project will not add to those
features, but it will make those features more publicly prominent and
bug-free.

\hypertarget{contact-information}{%
\subsection{Contact Information}\label{contact-information}}

\textbf{Name:} J. Chen

\textbf{Postal Address:} \texttt{not\ publicly\ available}.

\textbf{Telephone:} \texttt{not\ publicly\ available}.

\textbf{Email:} \texttt{not\ publicly\ available}.

\textbf{Other Communication Channels:} I'm fine communicating through
whatever means my mentors want. I have Slack, Telegram, and Signal, but
I have no problem with using another service.

\hypertarget{contributor-biography}{%
\subsection{Contributor Biography}\label{contributor-biography}}

I am currently an unaffiliated researcher, which is a fancy way of
saying that I do research without getting paid for it.\sidenote{\footnotesize One
  paper's just been submitted will hopefully be published this year!}
Until recently, I was a graduate student in psychology at the New School
for Social Research, though I also have an undergraduate background in
philosophy. I specialize in network psychometrics and substance use
disorders. I also have a strong interest in computational
psychopathology, and I'm hoping this project will let me focus on the
computational side for a few months.

I'm familiar with R, though mostly as a tool for research and data
analysis. I have no formal training in R and am entirely self-taught---I
imagine that this kind of programming background is not uncommon among
scientists. I'm sure this means that my knowledge of R has many holes,
even as I've worked to close them up. For this project, though, I think
it gives me a couple of advantages:

\begin{enumerate}
\def\labelenumi{\arabic{enumi}.}
\tightlist
\item
  Many of the scientists who are likely to use \texttt{animint2} have
  similar programming backgrounds, which I can take into account when
  documenting \texttt{animint2}.
\item
  I'm self-sufficient and used to resolving programming problems without
  the assistance of other people. Since I'll be working alone
  practically all of time, this will be a big help.
\end{enumerate}

\hypertarget{contributor-affiliation}{%
\subsection{Contributor Affiliation}\label{contributor-affiliation}}

No current affiliation. If my former affiliations matter: I worked and
studied at the New School for Social Research, and I was an extern
clinician at the Lower East Side Harm Reduction Center.

\hypertarget{schedule-conflicts}{%
\subsection{Schedule Conflicts}\label{schedule-conflicts}}

I'm not applying anywhere else, and I have no other conflicting
commitments (e.g.~I'm not travelling or anything). I have a weekly ASL
class that I attend, which may affect my weekly video call availability.

\hypertarget{mentors}{%
\subsection{Mentors}\label{mentors}}

\textbf{Evaluating Mentor:} Toby Dylan Hocking (tdhock5@gmail.com)

\textbf{Co-mentor:} Faizan Khan (faizan.khan.iitbhu@gmail.com)

\textbf{Have I been in contact?} Yes. I first contacted them via email
on either the 28th or the 29th of March, depending on the time zone.
I've maintained contact since.

\hypertarget{coding-plan-and-methods}{%
\subsection{Coding Plan and Methods}\label{coding-plan-and-methods}}

Two likely problems haunt projects like this one. Those problems are:
(a) the project will be delivered late, or (b) the project will not be
delivered at all. I've tried my best to address those problems by:

\begin{enumerate}
\def\labelenumi{\alph{enumi}.}
\tightlist
\item
  Ensuring that the scope of the project is limited and realistic, with
  allotted time slots for delays; and
\item
  Ensuring that the project is modular, which means a failure in one
  module will not affect other modules. Each module is either a week or
  a fortnight, with room for delays, and there are seven of them.
\end{enumerate}

For example: even if I don't manage to fix \texttt{animint2}'s layout
problem with multiple graphs, the reference website will be still be
there and totally usable. It's still one project with a united goal of
improved usability, but failure is now a matter of degree.

\begin{table}[b]
\begin{tabular}{@{}lll@{}}
\toprule
Module & Duration & When \\ \midrule
Reference website & Fortnight & Weeks 1 and 2 \\
Clarifying animint2 features in documentation & Fortnight & Weeks 3 and 4 \\
GitHub Issues completionist & Week & Week 5 \\
Fixing Chapter 7 & Week & Week 6 \\
Useful webpage for animint2 users & Week & Week 8 \\
Get rid of erroneous showSelect message & Week & Week 10 \\
Multiple graphs layout & Fortnight & Weeks 11 and 12 \\ \bottomrule
\end{tabular}
\end{table}

\pagebreak

\hypertarget{coding-timeline}{%
\subsection{Coding Timeline}\label{coding-timeline}}

\hypertarget{community-bonding-period-may-4-to-may-28}{%
\subsubsection{Community Bonding Period (May 4 to May
28)}\label{community-bonding-period-may-4-to-may-28}}

The goal for this bonding period is to (a) read through the entire
\href{https://rcdata.nau.edu/genomic-ml/animint2-manual/Ch00-preface.html}{\texttt{animint2}
Manual}, (b) refresh my knowledge of JavaScript, (c) improve my
knowledge of R, and (d) begin looking through the files in the
\texttt{animint2} repository:

\begin{enumerate}
\def\labelenumi{\alph{enumi}.}
\tightlist
\item
  I've read many chapters of the Manual already, but certainly not the
  whole thing. Since the \texttt{animint2} Manual is the key source of
  \texttt{animint2} information, I ought to study it carefully.
\item
  I'm thinking of reading some chapters of \emph{Eloquent JavaScript}
  and (if time allows) maybe doing a small JavaScript project.
\item
  I'll read through some chapters of \emph{Advanced R} (or all of it, if
  time allows). I mean to read through it all eventually---maybe that
  time is now.
\item
  Mostly I want to get a sense of what code \texttt{animint2} has
  borrowed from \texttt{ggplot2}, what code it's developed independent
  of \texttt{ggplot2}, and what code it's changed. I know that at least
  one bug is present in \texttt{animint2} that isn't in
  \texttt{ggplot2}.
\end{enumerate}

\hypertarget{week-1-may-29-to-june-4}{%
\subsubsection{Week 1 (May 29 to June
4)}\label{week-1-may-29-to-june-4}}

The goal this week is to reorganize files and get a draft of the
reference website ready.

\begin{itemize}
\tightlist
\item
  1st weekly meeting.
\item
  Work on a branch of
  \href{https://github.com/tdhock/animint2}{\texttt{animint2}}. Begin
  reorganizing files in a way that
  \href{https://pkgdown.r-lib.org/articles/pkgdown.html}{\texttt{pkgdown}
  recognizes}. \texttt{pkgdown} mostly wants a vignettes folder and a
  README that's a \texttt{.md} and not an \texttt{org}.
\item
  Compile preliminary \texttt{pkgdown} website.
\end{itemize}

\hypertarget{week-2-june-5-to-june-11}{%
\subsubsection{Week 2 (June 5 to June
11)}\label{week-2-june-5-to-june-11}}

The goal this week is to appropriately reorganize the functions, add
necessary links, and edit. Then I'll publish the website online.

\begin{itemize}
\item
  2nd weekly meeting.
\item
  Reorganize the \texttt{animint2} functions on the website.

  \begin{itemize}
  \tightlist
  \item
    By default, \texttt{pkgdown} organizes functions alphabetically, but
    it's possible to manually specify the order.
  \item
    My initial thought is to break up the functions by
    exclusivity---\texttt{animint2}-exclusive functions versus those
    also available in \texttt{ggplot2}.
  \end{itemize}
\item
  Ensure that the reference website has appropriate links to the
  \href{https://rcdata.nau.edu/genomic-ml/animint2-manual/Ch00-preface.html}{\texttt{animint2}
  Manual}, as well as vice versa.
\end{itemize}

\hypertarget{week-3-june-12-to-june-18}{%
\subsubsection{Week 3 (June 12 to June
18)}\label{week-3-june-12-to-june-18}}

The goal this week is to begin diagnosing and clearing up errors in both
the PDF and HTML documentation.

\begin{itemize}
\item
  3rd weekly meeting.
\item
  There are a number of functions in the
  \href{https://cran.r-project.org/web/packages/animint2/}{\texttt{animint2}
  reference} that just don't work. Test all the functions present in
  \texttt{animint2} and examine their code. Compile a list of functions
  that either just output a message or exhibit buggy behavior.

  \begin{itemize}
  \tightlist
  \item
    For example, \texttt{geom\_bin2d} is present, but when actually
    using it, \texttt{animint2} outputs: ``\texttt{bin2d} is not
    supported in \texttt{animint}. Try using \texttt{geom\_tile()} and
    binning the data yourself.'' This is confusing for the user, since
    \texttt{geom\_bin2d} is listed as one of \texttt{animint2}'s
    functions.
  \item
    Discuss this with my mentors. Toby has suggested either implementing
    functionality or simply removing them and their documentation.
  \end{itemize}
\end{itemize}

\hypertarget{week-4-june-19-to-june-25}{%
\subsubsection{Week 4 (June 19 to June
25)}\label{week-4-june-19-to-june-25}}

The goal this week is to continue diagnosing and clearing up errors in
the documentation.

\begin{itemize}
\item
  4th weekly meeting.
\item
  If necessary, continue compiling the list of problematic functions.
\item
  Either remove those functions from documentation, note (in the
  documentation) why they function they way they do, or mark them as
  bugs to be placed in GitHub Issues. The goal is not to fix them---I
  wouldn't have time for that.

  \begin{itemize}
  \tightlist
  \item
    It's possible that some functions will be straightforward to fix. If
    it takes a couple of hours to take something from \texttt{ggplot2}
    and implement it in \texttt{animint2}, and if I have some extra
    time, I imagine I might do it. That's up for discussion with my
    mentors.
  \end{itemize}
\end{itemize}

\hypertarget{week-5-june-26-to-july-2}{%
\subsubsection{Week 5 (June 26 to July
2)}\label{week-5-june-26-to-july-2}}

The goal this week is to document bugs on GitHub Issues, which will make
it easier to track and maintain.

\begin{itemize}
\tightlist
\item
  5th weekly meeting.
\item
  If I haven't done so already, add all the bugs noted in the earlier
  weeks and enter them as GitHub Issues.
\item
  Add all observed bugs from
  \href{https://gsoc.joss.cat/\#possible-bugs}{my website} to
  \texttt{animint2}'s issues.
\item
  Close or add bugs from the
  \href{https://github.com/tdhock/animint/issues}{\texttt{animint}
  repository} to \texttt{animint2}'s. I'll examine them to see if
  they're relevant anymore---at least
  \href{https://github.com/tdhock/animint/issues/150}{one is}.
\end{itemize}

\hypertarget{week-6-july-3-to-july-9}{%
\subsubsection{Week 6 (July 3 to July
9)}\label{week-6-july-3-to-july-9}}

The goal this week is to diagnose and repair Chapter 7 of the
\href{https://github.com/tdhock/animint-book}{\texttt{animint} Manual},
which scrolls up by itself.

\begin{itemize}
\tightlist
\item
  6th weekly meeting.
\item
  Figure out why Chapter 7 exhibits the auto-scrolling bug. (My guess is
  that it has something to do with the JavaScript.)
\item
  Repair the auto-scrolling error. Submit a pull request to the
  \texttt{animint2}-book repository.
\end{itemize}

\hypertarget{week-7-july-10-to-july-16}{%
\subsubsection{Week 7 (July 10 to July
16)}\label{week-7-july-10-to-july-16}}

It looks like Google
\href{https://developers.google.com/open-source/gsoc/timeline}{designates
this a non-coding period}. Midterm evaluations are due.

\begin{itemize}
\tightlist
\item
  7th weekly meeting.
\item
  Finish midterm evaluations by the 14th.
\end{itemize}

\hypertarget{week-8-july-17-to-july-23}{%
\subsubsection{Week 8 (July 17 to July
23)}\label{week-8-july-17-to-july-23}}

The goal this week is to add a vignette about common errors and
workarounds when using \texttt{animint2}, as well a vignette about any
major bugs to be aware of.

\begin{itemize}
\tightlist
\item
  8th weekly meeting.
\item
  Compile a list of those errors and workarounds (e.g.~``in lieu of
  \texttt{geom\_col()}, use \texttt{geom\_bar(stat\ =\ "identity")}'').
  Write a vignette addressing them and then add that to
  \texttt{vignette/}.
\item
  Compile a list of major bugs (e.g.~``it's not possible to layout
  multiple charts''). Write a vignette listing them, suggest
  workarounds, and then add that to \texttt{vignette/}.
\end{itemize}

\hypertarget{week-9-july-24-to-july-30}{%
\subsubsection{Week 9 (July 24 to July
30)}\label{week-9-july-24-to-july-30}}

The goal this week is to catch up on any missing work. This concludes
the larger documentation aspect of the project.

\begin{itemize}
\tightlist
\item
  9th weekly meeting.
\item
  Catch up on work, clean up messy code, edit documentation, and the
  like.
\end{itemize}

\hypertarget{week-10-july-31-to-august-6}{%
\subsubsection{Week 10 (July 31 to August
6)}\label{week-10-july-31-to-august-6}}

The goal this week is to figure out why using \texttt{showSelect} in
interactive scatterplots throws up an erroneous error, and then to fix
it.

\begin{itemize}
\item
  10th weekly meeting.
\item
  \texttt{animint()} incorrectly informs me that ``\texttt{showSelected}
  only works with \texttt{position=identity}, problem:
  geom1\_point\_foobar.''
\item
  Convert \texttt{showSelect} into a failing test, then repair the
  function so that the test passes.

  \begin{itemize}
  \tightlist
  \item
    I played around with \texttt{testthat} re: this error. Curiously,
    despite displaying a warning message, \texttt{expect\_warning} fails
    (and as a corollary, \texttt{expect\_no\_warning}
    succeeds).\sidenote{\footnotesize It's very possible that I made a mistake when
      generating my quick and dirty test.}
  \end{itemize}
\end{itemize}

\hypertarget{week-11-august-7-to-august-13}{%
\subsubsection{Week 11 (August 7 to August
13)}\label{week-11-august-7-to-august-13}}

The goal this week is to figure out why
\href{https://github.com/tdhock/animint/issues/150}{\texttt{animint2}
fails to correctly lay out multiple graphs}. I think this problem is
likely more complex than some other bugs, so I'll give myself two weeks
to figure it out.

\begin{itemize}
\item
  11th weekly meeting.
\item
  Figure out why the graphs aren't laid out correctly.

  \begin{itemize}
  \tightlist
  \item
    In my experience, \texttt{animint2} doesn't throw up any error
    messages---it just fails to correctly output the graphs.
  \item
    I wonder if the problem is not just with \texttt{animint2} but with
    how it interacts with other packages (or with Markdown, HTML, or
    \LaTeX).
  \item
    Toby mentioned that \texttt{animint()} just renders them one after
    another. In Quarto, when knit into a webpage's margins, they instead
    overlap with one another in such a way that no graph is usefully
    visible.
  \item
    \texttt{ggplot2} outputs multiple graphs correctly, but the problem
    was present when \texttt{animint2} depended on
    \texttt{ggplot2}---before \texttt{animint2} was a fork. It follows
    that there must be a problem with \texttt{animint2}-exclusive code,
    and not with \texttt{ggplot2}'s. Maybe that's a place to start.
  \end{itemize}
\end{itemize}

\hypertarget{week-12-august-14-to-august-20}{%
\subsubsection{Week 12 (August 14 to August
20)}\label{week-12-august-14-to-august-20}}

Having diagnosed the problem, the goal this week is to fix the problem
with multiple graph layouts.

\begin{itemize}
\item
  12th weekly meeting.
\item
  Convert \texttt{animint2} into a failing test, then repair the package
  so that the test passes.

  \begin{itemize}
  \tightlist
  \item
    Without knowing why the package fails to lay out the graphs, it's
    hard to explain what exactly I'll be doing.
  \item
    Another trouble here is that I can't just isolate a function to
    test.
  \item
    Toby Dylan Hocking suggests borrowing code from
    \texttt{flexdashboard}.
  \end{itemize}
\end{itemize}

\pagebreak

\hypertarget{week-13-august-21-to-august-28}{%
\subsubsection{Week 13 (August 21 to August
28)}\label{week-13-august-21-to-august-28}}

The goal this week is to catch up on unfinished work and submit mentor
evaluations. This concludes the smaller bug fix aspect of the project
and concludes the project altogether.

\begin{itemize}
\tightlist
\item
  13th weekly meeting.
\item
  Submit final mentor evaluation on the 28th.
\item
  Catch up on work, clean up messy code, edit documentation, and the
  like.
\end{itemize}

\hypertarget{management-of-coding-project}{%
\subsection{Management of Coding
Project}\label{management-of-coding-project}}

I expect to commit at least once a week, and almost certainly more. I
have a bad habit of committing exceedingly often at random
intervals.\sidenote{\footnotesize A likely artifact of being self-taught re: Git.} For
this project, I'll be more mindful of when and what I'm
committing---when some component of a module is started, serious
progress has been made, and that component is complete.

I'll also keep a public-facing notebook about what I've at done (and
what I'm struggling with), the techniques that I used, and resources
that I accessed. I'll publish at least once a week. That should keep me
accountable, prevent me from making the same mistake too many times, and
help me learn.

\hypertarget{tests}{%
\subsection{Tests}\label{tests}}

See the \href{https://gsoc.joss.cat}{\texttt{animint2} examination
website} for the tests that I took.

\hypertarget{anything-else}{%
\subsection{Anything Else?}\label{anything-else}}

Thanks to Toby Dylan Hocking for advice regarding changes to the easy
test, as well as advice regarding changes to this application.

Have a great day! \texttt{:\textgreater{}}



\end{document}
